\let\negmedspace\undefined
\let\negthickspace\undefined
\documentclass[journal,12pt,onecolumn]{IEEEtran}
\usepackage{cite}
\usepackage{amsmath,amssymb,amsfonts,amsthm}
\usepackage{algorithmic}
\usepackage{graphicx}
\usepackage{textcomp}
\usepackage{xcolor}
\usepackage{txfonts}
\usepackage{listings}
\usepackage{enumitem}
\usepackage{mathtools}
\usepackage{gensymb}
\usepackage{comment}
\usepackage[breaklinks=true]{hyperref}
\usepackage{tkz-euclide}
\usepackage{listings}
\usepackage{gvv}
\def\inputGnumericTable{}
\usepackage[latin1]{inputenc}
\usepackage{color}
\usepackage{array}
\usepackage{longtable}
\usepackage{calc}
\usepackage{multirow}
\usepackage{hhline}
\usepackage{ifthen}
\usepackage{lscape}

\newtheorem{theorem}{Theorem}[section]
\newtheorem{problem}{Problem}
\newtheorem{proposition}{Proposition}[section]
\newtheorem{lemma}{Lemma}[section]
\newtheorem{corollary}[theorem]{Corollary}
\newtheorem{example}{Example}[section]
\newtheorem{definition}[problem]{Definition}
\newcommand{\BEQA}{\begin{eqnarray}}
\newcommand{\EEQA}{\end{eqnarray}}
\newcommand{\define}{\stackrel{\triangle}{=}}
\theoremstyle{remark}
\newtheorem{rem}{Remark}
\begin{document}

\bibliographystyle{IEEEtran}
\vspace{3cm}

\title{EE1205: Signals and Systems - TA}
\author{EE22BTECH11039 - Pandrangi Aditya Sriram}
\maketitle

\renewcommand{\thefigure}{\theenumi}
\renewcommand{\thetable}{\theenumi}
\begin{enumerate}[label=\thechapter.\arabic*,ref=\thechapter.\theenumi]
\numberwithin{equation}{enumi}
\numberwithin{figure}{enumi}
\numberwithin{table}{enumi}
\item 
	The $Z$-transform of $p(n)$ is defined as
\begin{align}
P(z) = \sum_{n=-\infty}^{\infty}p(n)z^{-n}
\label{eq:ztrans}
\end{align}
\item If 
\begin{align}
	p(n) &= p_1(n)* p_2(n),
	\\
	P(z)&=P_1(z)P_2(z)
\end{align}
The above property follows from Fourier analysis and is fundamental to signal processing. 
\item For a Geometric progression 
\begin{align}
	x\brak{n} &=x\brak{0}r^nu\brak{n},
	\\
         \implies      X\brak{z} &= \sum_{n=-\infty}^{\infty}x\brak{n}z^{-n}
               =\sum_{n=0}^{\infty}x\brak{0}r^nz^{-n}\\
                &=\sum_{n=0}^{\infty}x\brak{0}\brak{rz^{-1}}^n\\
               &= \frac{x\brak{0}}{1-rz^{-1}}, \quad \abs{z}>\abs{r} 
	       \label{eq:gpz}
\end{align}
\item Let 
\begin{align}
	u(n) = 
	\begin{cases}
		1 & n \ge 0
		\\
		0 & \text{otherwise}
	\end{cases}
\end{align}
	       Substituting $r = 1$ in \eqref{eq:gpz},
\begin{align}
	u(n) \system{Z}	U(z) = 
                \frac{1}{1-z^{-1}}, \quad \abs{z}>1
	       \label{eq:uz}
\end{align}
\item From 
\eqref{eq:ztrans}
	       and 
	       \eqref{eq:uz},
\begin{align}
	U(z) &= \sum_{n = -\infty}^{\infty} u(n) z^{-n} 
	\\
\implies	\frac{dU(z)}{dz} &= -{z}^{-1} \sum_{n = -\infty}^{\infty} nu(n) z^{-n}\\
\therefore	nu(n) &\system{Z}\frac{z^{-1}}{(1-z^{-1})^{2}}, \quad \abs{z} > 1 
	       \label{eq:uzder}
\end{align}
\item For an AP, 
\begin{align}
	x(n) &= \sbrak{x(0) + nd} u(n) = x(0)u(n) + dnu(n)  \\
	\implies X(z) &= \frac{x(0)}{1-z^{-1}} + \frac{dz^{-1}}{(1-z^{-1})^{2}}, \quad \abs{z} > 1 
	       \label{eq:apz}
\end{align}
upon substituting from 
	       \eqref{eq:uz}
	       and
	       \eqref{eq:uzder}.
\item For the AP $x(n)$, the sum of first $n+1$ terms can be expressed as
\begin{align}
y(n) &= \sum_{k=0}^{n} x(k)\\
\implies y(n) &= \sum_{k=-\infty}^{\infty} x(k) u(n-k)\\
&= x(n) * u(n)
\end{align}
Taking the Z-transform on both sides, and substituting \eqref{eq:apz} and \eqref{eq:uz} (making use of the convolution property of Z-transform),
\begin{align}
Y(z) &= X(z) U(z)\\
\implies Y(z) &= \brak{\frac{x(0)}{1-z^{-1}} + \frac{dz^{-1}}{(1-z^{-1})^{2}}}\frac{1}{1-z^{-1}} \quad \abs{z} > 1 \\
&= \frac{x(0)}{({1-z^{-1}})^2} + \frac{dz^{-1}}{(1-z^{-1})^{3}}, \quad \abs{z} > 1 \\
&=x(0)\frac{z^2}{(z-1)^2} + \frac{dz^2}{(z-1)^{3}}, \quad \abs{z} > 1 \\
&=x(0)\brak{\frac{z}{(z-1)^2} + \frac{1}{z-1}} + \frac{d}{2}\brak{\frac{z^2 + z}{(z-1)^{3}}+\frac{z}{(z-1)^2}}, \quad \abs{z} > 1 \\
\end{align}
Taking the inverse Z-transform,
\begin{align}
y(n) &= \brak{x(0)(n+1) + \frac{d}{2}\brak{n^2 + n}} u(n)\\
&= \frac{n+1}{2}\brak{2x(0) + nd}u(n)\\
\implies \text{Sum }S(n) = y(n-1) &= \frac{n}{2}\brak{2x(0) + (n-1)d}, \quad n \geq 0\label{eq:apsum}
\end{align}
\end{enumerate}
\end{document}
