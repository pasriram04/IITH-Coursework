\documentclass[journal,12pt,onecolumn]{IEEEtran}
\usepackage{cite}
\usepackage{amsmath,amssymb,amsfonts,amsthm}
\usepackage{algorithmic}
\usepackage{graphicx}
\usepackage{textcomp}
\usepackage{xcolor}
\usepackage{txfonts}
\usepackage{listings}
\usepackage{enumitem}
\usepackage{mathtools}
\usepackage{gensymb}
\usepackage[breaklinks=true]{hyperref}
\usepackage{tkz-euclide} % loads  TikZ and tkz-base
\usepackage{listings}
\usepackage{float}


\begin{document}
\providecommand{\pr}[1]{\ensuremath{\Pr\left(#1\right)}}
\providecommand{\prt}[2]{\ensuremath{p_{#1}^{\left(#2\right)} }}        % own macro for this question
\providecommand{\qfunc}[1]{\ensuremath{Q\left(#1\right)}}
\providecommand{\sbrak}[1]{\ensuremath{{}\left[#1\right]}}
\providecommand{\lsbrak}[1]{\ensuremath{{}\left[#1\right.}}
\providecommand{\rsbrak}[1]{\ensuremath{{}\left.#1\right]}}
\providecommand{\brak}[1]{\ensuremath{\left(#1\right)}}
\providecommand{\lbrak}[1]{\ensuremath{\left(#1\right.}}
\providecommand{\rbrak}[1]{\ensuremath{\left.#1\right)}}
\providecommand{\cbrak}[1]{\ensuremath{\left\{#1\right\}}}
\providecommand{\lcbrak}[1]{\ensuremath{\left\{#1\right.}}
\providecommand{\rcbrak}[1]{\ensuremath{\left.#1\right\}}}
\newcommand{\sgn}{\mathop{\mathrm{sgn}}}
\providecommand{\abs}[1]{\left\vert#1\right\vert}
\providecommand{\res}[1]{\Res\displaylimits_{#1}} 
\providecommand{\norm}[1]{\left\lVert#1\right\rVert}
\providecommand{\norm}[1]{\lVert#1\rVert}
\providecommand{\mtx}[1]{\mathbf{#1}}
\providecommand{\mean}[1]{E\left[ #1 \right]}
\providecommand{\cond}[2]{#1\middle|#2}
\providecommand{\fourier}{\overset{\mathcal{F}}{ \rightleftharpoons}}
\newenvironment{amatrix}[1]{%
  \left(\begin{array}{@{}*{#1}{c}|c@{}}
}{%
  \end{array}\right)
}
\providecommand{\hilbert}{\overset{\mathcal{H}}{ \rightleftharpoons}}
\providecommand{\system}{\overset{\mathcal{Z}}{\longleftrightarrow}}
\newcommand{\solution}[2]{\textbf{Solution:}{#1}}
\newcommand{\cosec}{\,\text{cosec}\,}
\providecommand{\dec}[2]{\ensuremath{\overset{#1}{\underset{#2}{\gtrless}}}}
\newcommand{\myvec}[1]{\ensuremath{\begin{pmatrix}#1\end{pmatrix}}}
\newcommand{\mydet}[1]{\ensuremath{\begin{vmatrix}#1\end{vmatrix}}}
\newcommand{\myaugvec}[2]{\ensuremath{\begin{amatrix}{#1}#2\end{amatrix}}}
\providecommand{\rank}{\text{rank}}
\providecommand{\pr}[1]{\ensuremath{\Pr\left(#1\right)}}
\providecommand{\qfunc}[1]{\ensuremath{Q\left(#1\right)}}
	\newcommand*{\permcomb}[4][0mu]{{{}^{#3}\mkern#1#2_{#4}}}
\newcommand*{\perm}[1][-3mu]{\permcomb[#1]{P}}
\newcommand*{\comb}[1][-1mu]{\permcomb[#1]{C}}
\providecommand{\qfunc}[1]{\ensuremath{Q\left(#1\right)}}
\providecommand{\gauss}[2]{\mathcal{N}\ensuremath{\left(#1,#2\right)}}
\providecommand{\diff}[2]{\ensuremath{\frac{d{#1}}{d{#2}}}}
%\providecommand{\myceil}[1]{\left \lceil #1 \right \rceil }
\newcommand\figref{Fig.~\ref}
\newcommand\tabref{Table~\ref}
\newcommand{\sinc}{\,\text{sinc}\,}
\newcommand{\rect}{\,\text{rect}\,}
%%
%	%\newcommand{\solution}[2]{\textbf{Solution:}{#1}}
%\newcommand{\solution}{\noindent \textbf{Solution: }}
%\newcommand{\cosec}{\,\text{cosec}\,}
%\numberwithin{equation}{section}
%\numberwithin{equation}{subsection}
%\numberwithin{problem}{section}
%\numberwithin{definition}{section}
%\makeatletter
%\@addtoreset{figure}{problem}
%\makeatother

%\let\StandardTheFigure\thefigure
\let\vec\mathbf

\bibliographystyle{IEEEtran}

\vspace{3cm}

\title{EE1205: Signals and Systems - TA}
\author{EE22BTECH11039 - Pandrangi Aditya Sriram}
\maketitle

\renewcommand{\thefigure}{\theenumi}
\renewcommand{\thetable}{\theenumi}

\begin{enumerate}
%\begin{enumerate}[label=\thechapter.\arabic*,ref=\thechapter.\theenumi]
\numberwithin{equation}{enumi}
\numberwithin{figure}{enumi}
\numberwithin{table}{enumi}
\item 
	The $Z$-transform of $p(n)$ is defined as
\begin{align}
P(z) = \sum_{n=-\infty}^{\infty}p(n)z^{-n}
\label{eq:ztrans}
\end{align}
\item If 
\begin{align}
	p(n) &= p_1(n)* p_2(n),
	\\
	P(z)&=P_1(z)P_2(z)
\label{eq:prodz}
\end{align}
\item For a Geometric progression 
\begin{align}
	       \label{eq:gpn}
	x\brak{n} &=x\brak{0}r^nu\brak{n},
	\\
         \implies      X\brak{z} &= \sum_{n=-\infty}^{\infty}x\brak{n}z^{-n}
               =\sum_{n=0}^{\infty}x\brak{0}r^nz^{-n}\\
                &=\sum_{n=0}^{\infty}x\brak{0}\brak{rz^{-1}}^n\\
               &= \frac{x\brak{0}}{1-rz^{-1}}, \quad \abs{z}>\abs{r} 
	       \label{eq:gpz}
\end{align}
\item 	       Substituting $r = 1$ in \eqref{eq:gpz},
\begin{align}
	u(n) \system{Z}	U(z) = 
                \frac{1}{1-z^{-1}}, \quad \abs{z}>1
	       \label{eq:uz}
\end{align}
\item From 
\eqref{eq:ztrans}
	       and 
	       \eqref{eq:uz},
\begin{align}
	U(z) &= \sum_{n = -\infty}^{\infty} u(n) z^{-n} 
	\\
\implies	\frac{dU(z)}{dz} &= -{z}^{-1} \sum_{n = -\infty}^{\infty} nu(n) z^{-n}\\
\therefore	nu(n) &\system{Z}\frac{z^{-1}}{(1-z^{-1})^{2}}, \quad \abs{z} > 1 
	       \label{eq:uzder}
\end{align}
\item For an AP, 
\begin{align}
	x(n) &= \sbrak{x(0) + nd} u(n) = x(0)u(n) + dnu(n)  \\
	\implies X(z) &= \frac{x(0)}{1-z^{-1}} + \frac{dz^{-1}}{(1-z^{-1})^{2}}, \quad \abs{z} > 1 
	       \label{eq:apz}
\end{align}
upon substituting from 
	       \eqref{eq:uz}
	       and
	       \eqref{eq:uzder}.
\item From 
	\eqref{eq:conv-sum},  the sum to $n$ terms  of a GP can be expressed as
\begin{align}
	y(n) = x(n)*u(n) 
\end{align}
		where $x(n)$ is defined in 
	       \eqref{eq:gpn}.
From \eqref{eq:prodz}, \eqref{eq:gpz} and \eqref{eq:uz},
\begin{align}
	Y\brak{z}&= X\brak{z} U\brak{z}
	\\
	&=\brak{\frac{x\brak{0}}{1-rz^{-1}}}\brak{\frac{1}{1-z^{-1}}} \quad \abs{z} > \abs{r} \cap \abs{z}>\abs{1}
	\\
	&=\frac{x\brak{0}}{\brak{1-rz^{-1}}\brak{1-z^{-1}}} \quad \abs{z} > \abs{r} 
\end{align}
which can be expressed as
\begin{align}
	Y\brak{z}&=\frac{x\brak{0}}{r-1}\brak{\frac{r}{1-rz^{-1}} - \frac{1}{1-z^{-1}}}
\end{align}
using partial fractions.  Again, from \eqref{eq:gpz} and \eqref{eq:uz}, the inverse of the above can be expressed as
\begin{align}
	y\brak{n}
	&=x\brak{0}\brak{\frac{r^{n+1}-1}{r-1}}u\brak{n}
\end{align}

\item For the AP $x(n)$, the sum of first $n+1$ terms can be expressed as
\begin{align}
y(n) &= \sum_{k=0}^{n} x(k)\\
\implies y(n) &= \sum_{k=-\infty}^{\infty} x(k) u(n-k)\\
&= x(n) * u(n)
\end{align}
Taking the Z-transform on both sides, and substituting \eqref{eq:apz} and \eqref{eq:uz} (making use of the convolution property of Z-transform),
\begin{align}
Y(z) &= X(z) U(z)\\
\implies Y(z) &= \brak{\frac{x(0)}{1-z^{-1}} + \frac{dz^{-1}}{(1-z^{-1})^{2}}}\frac{1}{1-z^{-1}} \quad \abs{z} > 1 \\
&= \frac{x(0)}{({1-z^{-1}})^2} + \frac{dz^{-1}}{(1-z^{-1})^{3}}, \quad \abs{z} > 1 \label{eq:APSum}
\end{align}
But
\begin{align}
\frac{1}{({1-z^{-1}})^2} &= z\frac{z^{-1}}{({1-z^{-1}})^2}\\
\text{and w.k.t. } x(n+1) &\system zX(z), \quad \abs{z} > 1 \label{eq:time_shift}
\end{align}
From \eqref{eq:uzder} and \eqref{eq:time_shift}, 
\begin{align}
(n+1) u(n) &\system \frac{1}{({1-z^{-1}})^2}, \quad \abs{z} > 1
\end{align}
Also, w.k.t
\begin{align}
nx(n) &\system -z\frac{d}{dz}X(z)\\
\therefore \frac{z^{-1}}{(1-z^{-1})^{3}} = -\frac{z}{2}\frac{d}{dz}\sbrak{\frac{1}{(1-z^{-1})^{2}}} &\system \frac{1}{2}n\sbrak{(n+1)u(n)}
\end{align}
Therefore, taking inverse Z-transform of \eqref{eq:APSum}
\begin{align}
y(n) &= x(0)\sbrak{(n + 1)u(n)} + \frac{d}{2}\sbrak{n(n + 1)u(n)}\\
&= \frac{n+1}{2}\brak{2x(0) + nd}u(n)
\end{align}
\end{enumerate}
\end{document}